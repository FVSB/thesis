\documentclass[12pt,oneside]{uhthesis}
\usepackage{subfigure}
\usepackage[ruled,lined,linesnumbered,titlenumbered,algochapter,spanish,onelanguage]{algorithm2e}
\usepackage{amsmath}
\usepackage{amssymb}
\usepackage{amsbsy}
\usepackage{caption,booktabs}
\captionsetup{ justification = centering }
%\usepackage{mathpazo}
\usepackage{float}
\setlength{\marginparwidth}{2cm}
\usepackage{todonotes}
\usepackage{listings}
\usepackage{xcolor}
\usepackage{multicol}
\usepackage{graphicx}
\floatstyle{plaintop}
\restylefloat{table}
\addbibresource{Bibliography.bib}
% \setlength{\parskip}{\baselineskip}%
\renewcommand{\tablename}{Tabla}
\renewcommand{\listalgorithmcfname}{Índice de Algoritmos}
%\dontprintsemicolon
\SetAlgoNoEnd

\definecolor{codegreen}{rgb}{0,0.6,0}
\definecolor{codegray}{rgb}{0.5,0.5,0.5}
\definecolor{codepurple}{rgb}{0.58,0,0.82}
\definecolor{backcolour}{rgb}{0.95,0.95,0.92}

\lstdefinestyle{mystyle}{
    backgroundcolor=\color{backcolour},   
    commentstyle=\color{codegreen},
    keywordstyle=\color{purple},
    numberstyle=\tiny\color{codegray},
    stringstyle=\color{codepurple},
    basicstyle=\ttfamily\footnotesize,
    breakatwhitespace=false,         
    breaklines=true,                 
    captionpos=b,                    
    keepspaces=true,                 
    numbers=left,                    
    numbersep=5pt,                  
    showspaces=false,                
    showstringspaces=false,
    showtabs=false,                  
    tabsize=4
}

\lstset{style=mystyle}

\title{Un generador de problemas prueba para evaluar la calidad de la solución de los algoritmos de problemas de optimización de dos niveles }
\author{\\\vspace{0.25cm}Francisco Vicente Suárez Bellón}
\advisor{\\\vspace{0.25cm}Dr. C. Gemayqzel Bouza Allende}
\degree{Licenciado en Ciencia de la Computación}
\faculty{Facultad de Matemática y Computación}
\date{Fecha\\\vspace{0.25cm}\href{https://github.com/FVSB/Tesis}{github.com/FVSB/Tesis}}
\logo{Graphics/uhlogo}
\makenomenclature

\renewcommand{\vec}[1]{\boldsymbol{#1}}
\newcommand{\diff}[1]{\ensuremath{\mathrm{d}#1}}
\newcommand{\me}[1]{\mathrm{e}^{#1}}
\newcommand{\pf}{\mathfrak{p}}
\newcommand{\qf}{\mathfrak{q}}
%\newcommand{\kf}{\mathfrak{k}}
\newcommand{\kt}{\mathtt{k}}
\newcommand{\mf}{\mathfrak{m}}
\newcommand{\hf}{\mathfrak{h}}
\newcommand{\fac}{\mathrm{fac}}
\newcommand{\maxx}[1]{\max\left\{ #1 \right\} }
\newcommand{\minn}[1]{\min\left\{ #1 \right\} }
\newcommand{\lldpcf}{1.25}
\newcommand{\nnorm}[1]{\left\lvert #1 \right\rvert }
\renewcommand{\lstlistingname}{Ejemplo de código}
\renewcommand{\lstlistlistingname}{Ejemplos de código}

% NEW
% En el preámbulo del documento
\DeclareMathOperator*{\argmin}{argmin} % Define el operador argmin
\usepackage{cleveref} % En el preámbulo

\usepackage{array}

% Para que se ajusten automáticamente las tablas 
%\usepackage[utf8]{inputenc}
%\usepackage{geometry}
\usepackage{adjustbox}

% Hoja a4
%\geometry{a4paper, margin=1in}
% Hoja Carta
%\geometry{letterpaper, margin=1in}
\newenvironment{resultstable}[1]{
    \begin{table}[h!]
        \centering
        \caption{#1}
        \begin{adjustbox}{max width=\textwidth}
        \begin{tabular}{| l | l | l | l | l | l | l | }
            \hline
            \textbf{Tipo de punto} & \textbf{Nombre del problema} & \textbf{Punto estacionario} & \textbf{Valor objetivo del punto estacionario} & \textbf{Punto óptimo} & \textbf{Valor del punto óptimo} & \textbf{Método seleccionado}\\
            \hline
}{
        \end{tabular}
        \end{adjustbox}
    \end{table}
}

\newcommand{\resultrow}[7]{
    #1 & #2 & #3 & #4 & #5 & #6 & #7 \\
    \hline
}


% Poner los problemas binivel 
% Definición de entornos para problemas binivel
\usepackage{tcolorbox}

% Entorno principal para el problema binivel
% Entorno principal para el problema binivel
\newenvironment{bilevelmodel}[2] % Ahora acepta dos parámetros
{
    \begin{samepage}
    \begin{minipage}{\textwidth}
    \begin{center}
        \Large\textbf{Problema Binivel #1: #2} % #1 es el tipo, #2 es el nombre
    \end{center}
    \vspace{1em}
}
{
    \end{minipage}
    \end{samepage}
}
{}

% Entorno para el problema del líder
\newenvironment{upperlevel}[2][{}]
{
    \textbf{mín} \quad $#2$
    \vspace{0.5em}
    
    \textbf{sujeto a:}
    \ifx#1\empty
        % Si no se proporcionan restricciones, no mostrar align*
    \else
        \begin{align*}
            #1
        \end{align*}
    \fi
}
{}

% Entorno para el problema del seguidor
\newenvironment{lowerlevel}[2]
{
    %\begin{tcolorbox}[title=Problema del Nivel Inferior]
        \textbf{mín} \quad $#1$
        \vspace{0.5em}
        
        \textbf{sujeto a:}
        \begin{align*}
            #2
        \end{align*}
    %\end{tcolorbox}
}
{}

%Ejemplo
%\begin{bilevelmodel}{Lineal}{Nombre del Problema}
%    \begin{upperlevel}{F(x,y) = x + y}{
%        x + y \leq 10 \\
%        x, y \geq 0
%    }
%    \end{upperlevel}
%    
%    \begin{lowerlevel}{f(x,y) = 2x - y}{
%        2x + y \leq 8 \\
%        x, y \geq 0
%    }
%    \end{lowerlevel}
%\end{bilevelmodel}



\begin{document}

\frontmatter
\maketitle

\begin{dedication}
   



\end{dedication}
\begin{acknowledgements}

    \chapter*{Agradecimientos}

    El camino recorrido hasta este momento ha sido posible gracias al apoyo y dedicación de personas extraordinarias que han estado presentes en cada paso de mi formación académica. Deseo expresar mi más sincero agradecimiento a todos los profesores que han contribuido a mi desarrollo profesional durante mi trayectoria universitaria.
    
    Mi más profunda gratitud a la profesora Dra. C. Gemayqzel, mi tutora, por su paciencia, dedicación y capacidad para transmitir conocimientos en el fascinante mundo de la optimización binivel. Su apoyo y comprensión han sido fundamentales para la culminación exitosa de este trabajo.
    
    A mi familia, quiero agradecerles especialmente por su infinita paciencia y dedicación en los momentos más difíciles de este proceso. Su apoyo constante y su aliento para no abandonar nunca me han motivado a seguir adelante y alcanzar mis metas.
    
    A mis amigos y compañeros de estudio, gracias por su tiempo, dedicación y palabras de aliento, que enriquecieron esta experiencia académica. 
    
    Un agradecimiento especial a mis mascotas, por su compañía y alegría en los momentos desafiantes, y a ChatGPT, cuya asistencia fue invaluable en la redacción y organización de mis ideas.
    
    Este trabajo no habría sido posible sin el soporte y la guía de cada uno de ustedes. Gracias por ser parte fundamental de este capítulo de mi vida.
    
    
    
    
\end{acknowledgements}
\include{FrontMatter/SupervisorOpinion}
\begin{resumen}
	El problema de optimización binivel se define como minimizar una función sobre un conjunto determinado por los puntos óptimos de un modelo de programación matemática. La optimización en el nivel inferior depende de las decisiones tomadas en el nivel superior, creando así una relación de interdependencia entre ambos niveles.

	Para abordar este problema, se considera la creación de un problema relacionado en el cual el nivel inferior se sustituye por las condiciones necesarias de optimalidad, siendo este un problema con restricciones de complementariedad.

	En este trabajo se propone una forma de generar problemas de dos niveles cuyo problema con restricciones de complementariedad (MPEC) tiene un punto estacionario perteneciente a una de las siguientes 
	clases: fuertemente estacionario, M-estacionario o C-estacionario, dependiendo de los multiplicadores.

	\textbf{Palabras clave:} Optimización binivel, MPECs, Condiciones necesarias de optimalidad, Punto estacionario.
\end{resumen}

\begin{abstract}
	The bilevel optimization problem is defined as minimizing a function on a set determined by the optimal points of a mathematical programming model. The optimization at the lower level depends on the decisions made at the upper level, thus creating an interdependent relationship between both levels.

	To address this problem, the lower-level problem is replaced by the necessary optimality conditions and the resulting (relaxed) optimization problem with complementarity constraints is solved.
	
	This work proposes a way to generate two-level problems whose relaxed problem has a stationary point belonging to one of the following classes: strongly stationary, M-stationary, or C-stationary, depending on the multipliers.
	
	\textbf{Keywords:} Bi-level optimization, MPECs, Necessary optimality conditions, Stationary point.

\end{abstract}
\include{FrontMatter/Contents}

\mainmatter

\chapter*{Introducción}\label{chapter:introduction}
\addcontentsline{toc}{chapter}{Introducción}

%\section{Definición de problema de optimización binivel}
% Obtener una garantía de la solución es muy complejo (generalmente es estacionario y con puntos estacionarios).
La optimización de dos niveles, un área fundamental en la investigación operativa y la teoría de juegos, presenta desafíos significativos debido a su complejidad inherente. Este tipo de problemas se caracterizan por la interacción entre un líder y un seguidor, donde las decisiones del líder afectan las respuestas del seguidor. Uno de los aspectos más críticos de esta problemática es garantizar la existencia de soluciones óptimas, lo cual se ve complicado por la naturaleza no convexa del problema, incluso cuando las funciones y los conjuntos factibles son convexos. A menudo, los algoritmos utilizados en este contexto solo logran identificar puntos estacionarios o críticos, que no necesariamente representan soluciones locales o globales, ver \cite{DempeyZemkoho2020}.

% Definición de problema binivel
%\begin{equation}
%    \begin{aligned}
%    \text{minimizar} & \quad F(x, y) \\
%    \text{sujeto a} & \quad G(x, y) \geq 0 \quad (\text{restricciones de desigualdad}) \\
%    & \quad H(x, y) = 0 \quad (\text{restricciones de igualdad}) \\
%    & \quad y \in S(x) = \arg \min_{y} \{ f(x, y) \mid V(x, y) \geq 0, U(x, y) = 0 \}.
%    \end{aligned}
% \end{equation}

El modelo de dos niveles es:

\begin{equation*} 
    \begin{array}{l}
       \min_x \quad F(x, y)\\
        s.a \left\{ \begin{array}{l}
            x \in T \\
             y \in S(x) = \arg  \min_y \{ f(x, y) \quad s.a \quad y \in  H \}\\
            x,y \in M^0 
        \end{array}\right.
        \end{array} \end{equation*}

 \begin{samepage}
% Explicar dimensiones 
donde:

%Dimension de las variables
\begin{equation*}  
x \in R^{n}, \quad \quad y \in R^{m}
\end{equation*}
%Dimension de los niveles
\begin{align*}
    %Nivel superior
   & F : \mathbb{R}^{n} \times \mathbb{R}^{m} \to \mathbb{R}, 
   & \quad  T \subseteq \mathbb{R}^n ,
   \\
   % Nivel inferior
    & f : \mathbb{R}^{n} \times \mathbb{R}^{m} \to \mathbb{R} ,
    & \quad H \subseteq \mathbb{R}^m , 
\end{align*}
% Definir dimension de M^0
\begin{equation*}
    M^0 \subseteq \mathbb{R}^{n + m}
\end{equation*}
\end{samepage}
%Aclarar que la interacción jerárquica es compleja
En otras palabras, el problema de optimización binivel se centra en que el líder (nivel superior) debe tomar decisiones ($x$) que optimicen su objetivo $F(x, y)$, anticipando que el seguidor (nivel inferior) responderá de manera óptima con respecto a su propio objetivo $f(x, y)$, dado el valor de $x$ elegido por el líder. Esta interacción jerárquica entre ambos niveles añade una gran complejidad al problema en comparación con los problemas de optimización de un solo nivel.

%Sección de aplicaciones
%\section{Aplicaciones}
% Aplicaciones 
Los problemas de optimización de dos niveles son muy utilizados para modelar y analizar mercados eléctricos complejos, ofreciendo una perspectiva única sobre las interacciones estratégicas entre diversos agentes económicos.
% Mercado Eléctrico
% Electricity spot market with transmission losses
%En modelos como los que proponen en \cite{Aussel2013ElectricitySM} se tienen en cuenta las pérdidas de transmisión, esta contribución mejora significativamente la representación del sistema eléctrico, permitiendo un análisis más profundo del equilibrio estratégico mediante técnicas de \textbf{optimización}. Al considerar las pérdidas de transmisión, el modelo captura aspectos fundamentales de la distribución y comercialización de energía que anteriormente pasaban desapercibidos.
% Deregulated electricity markets with thermal losses and production bounds: models and optimality conditions
En el trabajo de  \cite{Aussel2016DeregulatedEM} se desarrolló un modelo innovador que aborda los mercados de electricidad desregulados. Su enfoque se distingue por incorporar restricciones de producción y pérdidas térmicas, lo que permite una modelización más precisa y realista. Mediante el uso de modelos binivel, los investigadores pueden explorar escenarios más complejos y representativos del funcionamiento real de los mercados energéticos.
% Nash equilibrium in a pay-as-bid electricity market Part 2 - best response of a producer Didier
Otro ejemplo en esta línea se encuentra en \cite{Aussel2017NashEI}, se profundiza en el análisis de mercados de electricidad de pago por oferta, explorando cómo un productor puede ajustar su estrategia considerando las acciones de sus competidores. El estudio destaca la aplicación de conceptos de \textbf{equilibrio de Nash} y técnicas de mejor respuesta, proporcionando una metodología sofisticada para optimizar la participación de un productor en el mercado.


% Machine Learning
Los modelos binivel también tiene aplicaciones fundamentales en la selección de hiperparámetros en aprendizaje automático, como lo demuestra el trabajo de \cite{DempeyZemkoho2020ML}. El capítulo 6 del libro aborda la optimización de hiperparámetros en problemas de clasificación y regresión, presentando algoritmos innovadores para manejar funciones objetivo no suaves y no convexas. La razón del uso de esta radica en su capacidad para minimizar errores en modelos complejos, mejorando así la precisión general del aprendizaje automático. Además, se implementan algoritmos especializados para abordar problemas no convexos.

% EPI
La optimización de dos niveles es una herramienta clave en el diseño y operación de redes industriales sostenibles. Ejemplos notables se incluyen
% Water integration in eco-industrial parks using a multi-leader-follower approach
en los estudios de \cite{Ramos2016WaterII} donde se optimiza el uso del agua a partir de modelar la situación mediante juegos de múltiples líderes-seguidores, priorizando objetivos ambientales y económicos. Los resultados mostraron que las empresas participantes lograron beneficios significativos con las formulaciones \textbf{KKT} (Karush-Kuhn-Tucker) del modelo \textbf{MLFG} (Multi-Leader-Follower Game) utilizado.
Además, en \cite{Ramos2016WaterII} se destaca la influencia de la estructura del juego en la configuración óptima, sugiriendo la necesidad de un diseño óptimo para cada planta dentro del EIP. 
Los enfoques \textbf{SLMFG} (Single-Leader-Multifollower Game) y \textbf{MLSFG} (Multi-Leader-Single-Follower Game) presentan variaciones en el rol de los participantes: en SLMFG, las empresas son seguidoras y la autoridad es líder, mientras que en MLSFG, ocurre lo contrario. Se resalta que el enfoque MLFG logra equilibrar objetivos económicos y ambientales, generando ahorros significativos mediante la reutilización de recursos. Los resultados indican una reducción en el consumo de agua fresca gracias a las estrategias implementadas, utilizando herramientas como GAMS para modelar los problemas de optimización, ver \cite{Ramos2016WaterII}. 
% Utility network optimization in eco-industrial parks by a multi-leader follower game methodology
Además, en \cite{Ramos2018UtilityNO} los autores introducen el concepto de autoridad ambiental en el diseño de redes de servicios públicos, utilizando juegos de múltiples líderes-seguidores y reformulaciones KKT. 
% Bi-level optimal low-carbon economic dispatch for an industrial park with consideration of multi-energy price incentive
En el ámbito del despacho energético bajo restricciones de carbono, en \cite{Gu2020BilevelOL} se modela incentivos de precios de energía en un parque industrial, demostrando que un enfoque binivel puede simultáneamente mejorar el impacto ambiental y los beneficios económicos, utilizando un procedimiento iterativo primal-dual.

% SLSF 
%A subsidy policy to managing hazmat risk in railroad transportation network
Además estudios como los de \cite{Bhavsar2021ASP} investigan sobre la aplicación de una política de subsidios para gestionar el riesgo de materiales peligrosos en una red de transporte ferroviario. En este modelo, el gobierno actúa como líder, ofreciendo subsidios para incentivar al operador ferroviario (el seguidor) a usar rutas alternativas que eviten los enlaces de alto riesgo en la red, utilizando el enfoque \textbf{SLSF} (Single-Leader-Single-Follower). Los autores utilizan una reformulación de KKT para resolver el problema y aplican su método a un caso real en los Estados Unidos. Se demuestra que incluso subsidios modestos pueden resultar en una reducción significativa del riesgo.


%\section{Dificultades teóricas y computacionales de los problemas binivel}
El estudio de los problemas de dos niveles es de interés de la comunidad científica no solo porque modelan las situaciones antes mencionadas, sino porque es complejo obtener propiedades de sus soluciones así como su cálculo numérico.
% Bilevel Difíciles
El concepto mismo de solución del problema binivel es complejo. La decisión del líder es solo respecto a un grupo de variables, mientras que las otras influyen en la función objetivo, pero no son decisión de él. Si para un mismo valor de las variables del líder el problema del seguidor tiene diferentes soluciones óptimas, el valor de la función objetivo del líder no estará determinado, sino que depende de cuál de los óptimos escogió el otro agente. 

%Condiciones Necesarias
% Sobre Valor extremal KKT y algoritmos 
Para obtener las condiciones de optimalidad y los algoritmos de solución de los modelos binivel se reportan dos enfoques fundamentales. En \cite{DempeyZemkoho2020} se usa la función valor extremal. 
%Explicación sobre que va el valor extremal
Esto significa que para todo valor de $x$, se considera la minimización de la función objetivo del líder en el conjunto dado por las restricciones de ambos individuos y la condición de que la función objetivo del seguidor es menor o igual que el valor más pequeño que alcanza en el conjunto de soluciones factibles del seguidor.
 
% Al realizar el kkt en el level inferior se transforma a un MPEC
Otro enfoque clásico consiste en sustituir el problema del nivel inferior por la condición de KKT, lo que permite transformar problemas de optimización binivel en programas matemáticos con restricciones de equilibrio (MPEC), facilitando así su resolución, ver \cite{AnnotatedBibliographyDempe,Caselli2024BilevelOW,DempeyZemkoho2020}.
Conocido como \textit{enfoque KKT} en la literatura, es una de las formas más utilizada para la resolución de problemas de dos niveles en la actualidad, ver \cite{Aussel2021GenericityAO}.


% NP-Hard
% The polynomial hierarchy and a simple model for competitive analysis
% Some properties of the bilevel programming problem
Dado que los problemas de optimización de ese tipo son inherentemente difíciles de resolver debido a su naturaleza \textbf{NP-hard}, ver \cite{Jeroslow1985ThePHNP,jonathan_f__bard_1991NP}
% Sigma P2Hard
% Libro Dempe
% Tesis doctoral Cerulli
o incluso $\Sigma P2-hard$, ver \cite{phdthesisCerulli, DempeyZemkoho2020},
se han desarrollado diversos enfoques para abordar su complejidad computacional. 
Sin embargo, estos enfoques suelen ser computacionalmente intensivos para problemas de gran escala, ver \cite{phdthesisCerulli}.
%A Review on Bilevel Optimization: From Classical to Evolutionary Approaches and Applications
En paralelo, los algoritmos metaheurísticos, como los evolutivos, han ganado relevancia al proporcionar aproximaciones eficientes en casos no lineales o no convexos, donde las soluciones exactas son inalcanzables en tiempos razonables, ver \cite{Sinha2017ARO}.
% Dividir en subproblemas y resolver iterativamente
% Collection of Test Problems for Constrained Global Optimization Algorithms
Otro enfoque destacado es el uso de métodos de descomposición, los cuales dividen el problema en subproblemas más manejables que pueden resolverse iterativamente, ver \cite{Floudas1990ACO}.

% An SOS1-based approach for solving MPECs with a natural gas market application
Estas técnicas son particularmente útiles en aplicaciones prácticas, como los mercados de energía o los modelos de sostenibilidad, ver \cite{SadddiquiNaturalGasSOS1}.
% Libro Dempe
A pesar de estos avances, existen desafíos abiertos. La escalabilidad sigue siendo un problema crítico, ya que el crecimiento exponencial de las opciones en problemas de gran tamaño limita la aplicabilidad de los métodos exactos, ver \cite{DempeyZemkoho2020}. Asimismo, los problemas no convexos carecen de garantías de convergencia hacia el óptimo global, lo que los hace especialmente difíciles de abordar. Finalmente, la incorporación de incertidumbre en los modelos agrega una capa adicional de complejidad, lo que demanda nuevos enfoques híbridos que combinen algoritmos exactos y heurísticos para mejorar la eficiencia computacional sin sacrificar la calidad de las soluciones, ver \cite{phdthesisCerulli,Sinha2017ARO}. Estos avances y desafíos reflejan la importancia de diseñar algoritmos personalizados que aprovechen las estructuras particulares de cada problema binivel. Las aplicaciones industriales, como el diseño de redes ecoindustriales y la gestión de mercados energéticos, destacan la necesidad de enfoques que equilibren precisión y tiempo de cálculo, haciendo de la optimización de dos niveles un área de investigación activa con un impacto significativo en la práctica.



% Métodos matemáticos que se utilizan KKT y branch and bround
Los algoritmos que se basan en las condiciones KKT incluyen una variedad de métodos, como técnicas de branch-and-bound y métodos de suavizado, ver \cite{DempeyZemkoho2020}. 
% Hablar sobre el Suavizado KKT y Del SQP ademas de que el KKT se extiende a algoritmos evolutivos
Además, se menciona que los algoritmos SQP (Sequential Quadratic Programming) también se fundamentan en las condiciones KKT para resolver problemas suaves con restricciones. 
Este hecho demuestra su versatilidad diversas áreas de la optimización.

% Que son no convexos por las transformaciones
Los problemas de optimización de dos niveles y la formulación KKT son inherentemente no convexos, lo que implica que los métodos de optimización convexa no son directamente aplicables. Esta no convexidad puede llevar a soluciones subóptimas y a dificultades para encontrar una solución global. 
Aunque en muchos casos las funciones involucradas en la definición del modelo sean convexas, la estructura general del problema sigue siendo no convexa.

En resumen, obtener garantías sobre soluciones en problemas de optimización de dos niveles es un desafío complejo debido a que por su no convexidad inherente, se presentan dificultades para escapar de óptimos locales, la posible falta de unicidad y la inestabilidad en las soluciones. Los algoritmos frecuentemente hallan puntos estacionarios, que no siempre corresponden a las soluciones óptimas deseadas. Por lo tanto, es crucial desarrollar métodos especializados que aborden estos problemas y permitan encontrar soluciones globales o aproximaciones adecuadas, ver \cite{DempeyZemkoho2020}.

% Se añadió la genericidad
% Paper de gema sobre  los puntos estacionarios
En este contexto, se ha estudiado la estructura genérica de los problemas de complementariedad (MPCC) que surgen del enfoque KKT y Fritz-John (FJ) aplicado a un problema de dos niveles. Se ha demostrado que, para una clase amplia de estos, la condición de que independencia lineal (MPCC-LICQ) se cumple en todos los puntos factibles. Sin embargo, las condiciones de complementariedad estricta (MPCC-SC) y las condiciones de segundo orden (MPCC-SOC) pueden fallar en puntos críticos (estacionarios), incluso en situaciones genéricas, ver \cite{Allende2012SolvingBP}. Esta situación complica aún más la obtención de garantías sobre la solución.

Es importante señalar que existen casos singulares donde los puntos estacionarios pueden ser problemáticos, especialmente cuando el multiplicador (\(\alpha\)) asociado a la condición de KKT del problema de nivel inferior es igual a cero. En tales circunstancias, la condición MPCC-SC puede no cumplirse, lo que podría llevar a que el método KKT no funcione adecuadamente, ver \cite{Allende2012SolvingBP}.

% Qué se va a hacer en la tesis
Basándose en la caracterización de los diferentes tipos de puntos estacionarios establecida por \cite{Flegel2003AFJ}, esta tesis propone desarrollar un generador de problemas que, dado un punto y las funciones $F$, $g_s$, $f$ y $v_s$  que definen un problema de dos niveles, agregarles funciones polinomiales de primer o segundo grado de forma tal que el punto inicial dado sea un punto crítico del problema creado. 
Este generador facilitará el estudio del comportamiento de algoritmos conocidos en problemas con ahora al menos un punto estacionario conocido.
El usuario además podrá decidir si quiere un punto crítico con multiplicadores arbitrarios o si $\alpha = \vec{0}$, lográndose estudiar las clases de puntos críticos que aparecen en el caso genérico, o sea en un clase amplia y significativa de los problemas generados.

%NOTA:
% Clase significativa y amplia:
% Es que todo problema es límite de problemas de la clase,
% y si un problema está en la clase,
% para toda perturbación de clase C3 suficientemente pequeña,
% el problema perturbado sigue estando en la clase.


% Explicación 
% 2do cap
La tesis está compuesta de 3 capítulos luego del capítulo de introducción, en un segundo capítulo 
se mostrará la notación que se empleará, se define el problema de dos niveles con un líder y un seguidor, y se explica la teoría matemática para su transformación en un problema MPEC, así como los algoritmos de Julia que se utilizarán en ella.
% 3ero
En el tercer capítulo se explicará la implementación algorítmica propuesta anteriormente y su correcta utilización. 
% 4to
En el cuarto capítulo se analizarán los resultados obtenidos por el algoritmo propuesto y su comparación con algoritmos implementados en el entorno Julia.
% Siguiente
Finalmente, se presentarán las conclusiones y recomendaciones del trabajo realizado.
\chapter{Preliminares}\label{chapter:state-of-the-art}



La optimización binivel es un problema de optimización donde un subconjunto de variables debe ser la solución óptima de otro problema de optimización, parametrizado por las variables restantes. Este problema tiene dos niveles jerárquicos de decisión: el nivel superior (líder) y el nivel inferior (seguidor).
Este tiene dos características principales: primero, el problema del nivel inferior actúa como una restricción para el nivel superior; segundo, la solución del nivel inferior depende de las variables del nivel superior, creando una interdependencia entre ambos niveles. Por ello, el líder debe anticipar la respuesta óptima del seguidor al tomar decisiones.
%Explicación breve de binivel
En términos abstractos, la optimización binivel busca minimizar una función objetivo de nivel superior, $F(x, y)$, donde $x$ son las variables de decisión del líder y $y$ son las variables del seguidor. 

% Explicación del capitulo
En este capítulo se abordará los conocimientos necesarios para el desarrollo de esta tesis. Consta de secciones sobre los conceptos fundamentales de la Optimización Binivel del Caso Optimista, así como su reformulación KKT,
además de nociones clave sobre los Problemas Matemáticos con Restricciones de Equilibrio (MPEC), 
y por último la modelación de los problemas binivel en el lenguaje de programación Julia y métodos seleccionados implementados en este lenguaje para su resolución. 

\section{Optimización Binivel }
%Intro de la sección
El problema de optimización binivel en general se define de la siguiente manera
\begin{equation} \label{eq:Def1Binivel}
    \begin{array}{l}
       \min_{x} \quad F(x, y)\\
        s.a \left\{ \begin{array}{l}
            x \in T \\
             y \in S(x) = \arg  \min_{y} \{ f(x, y) \quad s.a \quad y \in  \cal{H}\\
            x,y \in M^0 
        \end{array}\right.
        \end{array} \end{equation}
En la mayoria de las aplicaciones, los problemas de optimizacion del nivel inferior y superior son modelos de programacion matematica, por lo  que tienen la estructura  *** con S(x)
\begin{equation}
\begin{aligned}
& \min_{x} \; F(x, y) \\
& \text{sujeto a } \\
& G(x) \leq 0, \\
& y \in \argmin_{y} \left\{ f(x, y) \mid v(x, y) \leq 0 \right\}
\end{aligned}
\label{eq:ProblemaGeneral}
\end{equation}
Para reducir las notaciones asumimos que en ambos problemas solo hay restricciones de desigualdad. O sea, se considerara el modelo 
\begin{equation}
\begin{aligned}
& \min_{x} \; F(x, y) \\
& \text{sujeto a } \\
& G(x) \leq 0, \\
& y \in \argmin_{y} \left\{ f(x, y) \mid v(x, y) \leq 0 \right\}
\end{aligned}
\label{eq:ProblemaGeneral}
\end{equation} 
%TODO:
% ***CON S(x)
%Para el concepto de solucion es necesario tener en cuenta la decision de ... parrafo de intro cambiado
%En este sentido, se consideran dos enfoques principales: el optimista y el pesimista.  El problema optimista consiste en  y se 
%% Formulación general de optimización binivel


% DIFERENCIA OPTIMISTA PESIMISTA
% Decir que en el libro de Dempe se hablan de estos dos enfoques
%Enfoque optimista
En el enfoque optimista, se asume que el seguidor, que actúa en el nivel inferior, elegirá la solución más favorable para el líder, quien toma decisiones en el nivel superior. Este es considerado más tratable y, en ciertas situaciones favorables, puede simplificarse a un problema convexo. Además, en el contexto de múltiples objetivos, el enfoque optimista permite alcanzar el mejor frente de Pareto posible, ver \cite{DempeyZemkoho2020}.

%\begin{center}   
%    Bajo el enfoque optimista :
%    \end{center}
    % Enfoque optimista (selección favorable)
    \begin{equation}
    \begin{aligned}
    & \min_{x} \; \inf_{y \in S(x)} F(x, y) \\
    & \text{con } S(x) = \argmin_{y} \left\{ f(x, y) \mid v(x, y) \leq 0 \right\}
    \end{aligned}
    \label{eq:EnfoqueOptimistaGeneral}
    %\caption*{Problema de optimización bajo el enfoque optimista}
    \end{equation}
    \captionof{figure}{Problema de optimización bajo el enfoque optimista.}

% Enfoque pesimista
Por otro lado, el enfoque pesimista asume que el seguidor seleccionará la opción menos favorable para el líder entre las soluciones óptimas disponibles, el cual es más complejo de resolver y puede incluso no tener solución. A menudo, se requieren reformulaciones para abordar estos problemas, lo que lo convierte en un reto teórico y computacional significativo en situaciones de múltiples objetivos, conduce al peor frente de Pareto posible, ver \cite{Sinha2017ARO}.
%\begin{center}
%    Bajo el enfoque pesimista:
%\end{center}
% Enfoque pesimista (peor caso)
\begin{equation}
\begin{aligned}
& \min_{x} \; \sup_{y \in S(x)} F(x, y) \\
& \text{con } S(x) = \argmin_{y} \left\{ f(x, y) \mid v(x, y) \leq 0 \right\}
\end{aligned}
\label{eq:EnfoquePesimistaGeneral}
\end{equation}
\captionof{figure}{Problema de optimización bajo el enfoque pesimista.}
%Decir que el optimista es mejor
Es relevante destacar que la mayoría de la literatura se centra en el enfoque optimista debido a su mayor facilidad de tratamiento. Sin embargo, el otro también tiene su utilidad, especialmente en la modelación de situaciones donde se considera la aversión al riesgo, ver \cite{DempeyZemkoho2020}. En este contexto, los términos ''líder'' y ''seguidor'' se utilizan para describir los roles en el modelo a optimizar; el líder toma decisiones considerando las posibles reacciones del seguidor, quien a su vez reacciona seleccionando su mejor opción, ver \cite{Sinha2017ARO}.

% Decir que se va ha hablar del enfoque optimista
Dado que en la tesis trataremos sobre problemas binivel de enfoque optimista mostraremos algunos resultados referidos al modelo resultante.


\newpage
% Definición de problema binivel

\textbf{Optimización Binivel Optimista.} Un problema de optimización binivel optimista, se formula como:

\begin{align}
    \min_{x , y} & \quad F(x, y) \notag \\
    \text{s.t.} & \quad g_i(x, y) \leq 0, i=1\ldots q,  \notag \\
    & \quad h_i(x,y) = 0, i=1 \ldots r, \notag\\
    & \quad y \in S(x), \notag\\
    &\intertext{donde $S(x)$ es el conjunto de soluciones óptimas del problema parametrizado por $x$} \tag{\theequation}\\
    \min_{y \in Y(x)} & \quad f(x, y) \notag \\
    \text{s.t.} & \quad v_j(x, y) \leq 0, j=1\ldots s, \notag\\
    & \quad u_j(x,y) = 0, j=1 \ldots t, \notag\\
    %&\text{Definición de un Problema Binivel Optimista} \notag\\
\end{align}
\label{eq:DefBinivelOptimista} 
%\textbf{Donde $M^0(x,y)$ es el conjunto de restricciones comunes para ambos niveles}

\begin{samepage}
%Dimensiones
donde 
% Dimension de x and y
\begin{equation*}
    x \in R^{n},\quad \quad y \in R^{m}
\end{equation*}
\begin{align*}
    %Dimension de F 
& F(x,y) : \mathbb{R}^{n} \times \mathbb{R}^{m} \to \mathbb{R},
%Dimension g_i
& \quad  g_i(x,y)  : \mathbb{R}^{n} \times \mathbb{R}^{m} \to \mathbb{R} ,
%Dimension h_i
& \quad h_i(x,y)  : \mathbb{R}^{n} \times \mathbb{R}^{m} \to \mathbb{R} ,
\\
% Def nivel inferior
%Dimension de f_i
&f(x,y) : \mathbb{R}^{n} \times \mathbb{R}^{m} \to \mathbb{R},
% Dimension de v_j
& \quad v_j(x,y)  : \mathbb{R}^{n} \times \mathbb{R}^{m} \to \mathbb{R} ,
% Dimension u_j
& \quad u_j(x,y)  : \mathbb{R}^{n} \times \mathbb{R}^{m} \to \mathbb{R} ,
\end{align*} 
\end{samepage}

%\begin{equation*}
%    %Dimension de M_i
%m_i \in M^0(x,y) \| m_i : f \mathbb{R}^{n} \times \mathbb{R}^{m} \to \mathbb{R}
%\end{equation*}

% Esto dejarlo en una sola pagina
%TODO: RESTRICCIONES DEL LIDER Y DEL SEGUIDOR NO IMPLICITA y EXPLICITA
 %TODO: RESTRICCIONES DEL LIDER Y DEL SEGUIDOR NO IMPLICITA y EXPLICITA
\newpage


Esta minimización está sujeta a dos tipos de restricciones: 
%Restricciones explicación old
%las restricciones explícitas para el líder, $x \in X$, donde $X$ es el conjunto de valores factibles para las variables del líder; y las restricciones implícitas impuestas por el seguidor, donde $y$ debe pertenecer al conjunto de soluciones óptimas del problema de optimización del seguidor, $\arg\min\{f(x, y) : y \in Y(x)\}$. En este contexto, $f(x, y)$ es la función objetivo del nivel inferior, y $Y(x)$ representa las restricciones del nivel inferior, las cuales pueden depender de las variables de decisión del líder, $x$.
% Restricciones nuevas
\begin{itemize}
    \item \textbf{Restricciones explícitas:}
    \begin{itemize}
        \item Desigualdades del líder: \( g_i(x,y) \leq 0 \quad \forall i \)
        \item Igualdades del líder: \( h_i(x,y) = 0 \quad \forall i \)
    \end{itemize}
    
    \item \textbf{Restricciones implícitas:}
    \begin{itemize}
        \item Impuestas por el seguidor: \( y \in \arg\min\{f(x, y) : v_j(x,y) \leq 0,\ u_j(x,y) = 0\} \),\\
        donde \( f(x, y) \) es la función objetivo del seguidor
    \end{itemize}
    
	 
	\item También se puede asumir que ambas variables tienen restricciones conjuntas, las restricciones $ g_i(x,y), \quad h_i(x,y), \quad v_j(x,y), \quad u_j(x,y) $ que dependen de las variables $x$ y $y$.
	
	
\end{itemize}
 % Hablar sobre el enfoque optimista
%En el caso del enfoque optimista si para el nivel inferior existen más de un punto que resuelve el problema del nivel inferior y tomará la que más beneficie al nivel superior. 
 % Decir que solo restricciones desigualdad
Por simplicidad solo se consideran restricciones de desigualdad.


%\subsection{Transformación de los problemas de dos niveles}
		
		Los problemas de dos niveles pueden ser reformulados en un problema de un solo nivel al reemplazar el problema del nivel inferior por las condiciones KKT de este en las restricciones del primer nivel. 
		
		%Para el caso de los SLSMG donde se tiene un problema de optimización en el nivel inferior este sustituye por de las condiciones KKT de este, obteniendo un MPEC \autocite{aussel2020}.
        
% Descripción del modelo en KKT
       \begin{table}[H]
        \centering
        \begin{equation}
            \begin{array}{l}
                \underset{\substack{x, y, \lambda_j}}{\min} \quad F(x, y)\\
                s.a \left\{ 
                \begin{array}{l}
                    g_i(x, y) \leq 0, i=1\ldots q,\\
                    \nabla_{y} f(x, y) + \sum_{j=1}^{s} \nabla_{y} v_j(x, y) \lambda_j = 0, \\
                    v_j(x, y) \leq 0, j=1\ldots s,\\
                    v_j(x, y)\lambda_j = 0, j=1\ldots s, \\
                    \lambda_j \geq 0, j=1\ldots s\\
                \end{array}\right.
                \tag{\theequation}
            \end{array}
            \label{eq:KKT_Optimista}
        \end{equation}
        \caption*{MPEC resultante}

    \end{table}
    
% Añadir aclaratoria
Los tres últimos grupos de restricciones expresan que $v$ y $\lambda$ están restringidas en signo y que al menos una es 0. Estas condiciones son conocidas como \textbf{restricciones de complementariedad}. Estos modelos corresponden a la clase de problemas de programación matemática con restricciones de complementariedad (MPEC). A continuación, presentamos los resultados de esta área necesarios para el desarrollo de esta tesis.
%TODO: Como se dijo en la intro existe otra forma, en un problema de programacion matematica  usando funcion del valor extremal sin embargo salvo en casos muy particulares esta no tiene una expresion explicita. La ventaja dsel kkt por mpec es que es una forma explicita, pero se requiere algun tipo de condicion de regularidad del conjunto como LICQ independencia  de v activa, que caso ded no cumpliser poder existir puntos optimos que no lo recoga el kkt. 

\section{Sobre los Programas Matemáticos con Restricciones de Equilibrio (MPEC)}
%Aca va la sección que explica los MPEC
% Que es un MPEC
Un \textbf{Programa Matemático con Restricciones de Equilibrio (MPEC)} es un tipo de programa no lineal que incluye restricciones de equilibrio, específicamente, restricciones de complementariedad.

%Formula MPEC Genérico
\begin{equation}
\begin{aligned}
\min \quad & f(z)  \\
\text{s.t.} \quad & g(z) \leq 0, \quad h(z) = 0 \\
& G(z) \geq 0, \quad H(z) \geq 0, \quad G(z)^T H(z) = 0 \\
&\text{Definición de MPEC} \\
\end{aligned}  
%\tag{\theequation} 
\label{eq:DefMpec}
\end{equation}

%Dimensiones MPEC Genérico
donde $f: \mathbb{R}^n \to \mathbb{R}$, $g: \mathbb{R}^n \to \mathbb{R}^m$, $h: \mathbb{R}^n \to \mathbb{R}^p$, $G: \mathbb{R}^n \to \mathbb{R}^\ell$, y $H: \mathbb{R}^n \to \mathbb{R}^\ell$ son funciones continuamente diferenciables. Debido al término de complementariedad en las restricciones, los programas de este tipo son algunas veces referidos como programas matemáticos

Estas restricciones de complementariedad se expresan como: 

%Ecuación para poner que son las restricciones de complementariedad
\begin{equation}
    G(z) \geq 0, \quad H(z) \geq 0, \quad \text{y} \quad G(z)^T H(z) = 0 \label{eq:RestriccionesComplementariedadAbstracto}
\end{equation}

Los MPEC incluyen restricciones donde el producto de dos funciones (\textit{G} y \textit{H}) debe ser cero, y ambas funciones deben ser no negativas. Esto se conoce como restricción de complementariedad. Debido a las restricciones de complementariedad, los MPEC no cumplen con las condiciones de regularidad estándar, lo que hace que las condiciones de KKT no sean directamente aplicables como condiciones de optimalidad de primer orden. Los MPEC son utilizados para modelar problemas donde existen restricciones de equilibrio, como problemas de ingeniería y economía, ver \cite{Flegel2003AFJ,DempeyZemkoho2020}.


%\subsection{Resultados sobre MPECs}

%Notación de los indices activos del documento
A continuación como en \cite{Flegel2003AFJ}, se introduce las siguientes definiciones.
\begin{definition}[Conjunto de Índices]
Dado un vector factible $z^*$ del MPEC \eqref{eq:DefMpec}, definimos los siguientes \textit{conjuntos de índices}:
\begin{equation}
\begin{aligned}
J_G &:= J_G(z^*) := \{i|G_i(z^*) = 0, H_i(z^*) > 0\} \\
J_{GH}:= J_{GH}:(z^*) := \{i|G_i(z^*) = 0, H_i(z^*) = 0\}  \\
J_H &:= J_H(z^*) := \{i|G_i(z^*) > 0, H_i(z^*) = 0\}  \\
\end{aligned}
\label{eq:ConjuntoDeIndices} 
\end{equation}
\end{definition}

% Definición de TNLP Problema No Lineal Ajustado
Para definir calificaciones de restricción alteradas, introducimos el siguiente problema, dependiente de $z^*$:

\begin{definition}[Programa No Lineal Ajustado (TNLP)]

Un \textit{Programa No Lineal Ajustado $TNLP := TNLP(z^*)$} es:
%TODO: INcomporar indices y rango en que se mueven
\begin{equation}
\begin{aligned} 
\min \quad & f(z) \\
\text{s.t.} \quad & g_{i}(z) \leq 0 \quad h(z) = 0  \\
& G_{\alpha \cup \beta}(z) = 0 \quad G_{\gamma}(z) \geq 0  \\
& H_{\alpha}(z) \geq 0 \quad H_{\gamma \cup \beta}(z) = 0  \\
& \text{Problema No Lineal Ajustado (TNLP)}  
\end{aligned}
\label{eq:ProblemaAbstractoTNLP}
\end{equation}
    
\end{definition}

El TNLP \eqref{eq:ProblemaAbstractoTNLP} puede ahora ser usado para definir variantes MPEC adecuadas de las calificaciones de restricción estándar de independencia lineal, Mangasarian-Fromovitz y Mangasarian-Fromovitz estricta (LICQ, MFCQ, y SMFCQ en su forma abreviada).

\begin{definition}[MPEC-LICQ]
El MPEC \eqref{eq:DefMpec} se dice que satisface la \textit{MPEC-LICQ} (MPEC-MFCQ, MPEC-SMFCQ) en un vector factible $z^*$ si el correspondiente TNLP$(z^*)$ satisface la LICQ (MFCQ, SMFCQ) en ese vector $z^*$.
\end{definition}
    
% Hacer notar que un mínimo de z^* implica la existencia de \lambda^* ....
%En este punto, es importante notar que bajo MPEC-MFCQ, un mínimo local $z^*$ del MPEC \eqref{eq:DefMpec} implica la existencia de un multiplicador de Lagrange $\lambda^*$ tal que $(z^*, \lambda^*)$ satisface las condiciones KKT para el programa \eqref{eq:ProblemaAbstractoTNLP}. Por lo tanto, si asumimos que MPEC-MFCQ se cumple para un mínimo local $z^*$ del MPEC \eqref{eq:DefMpec}, podemos usar cualquier multiplicador de Lagrange $\lambda^*$ (que ahora se sabe que existe) para definir el MPEC-SMFCQ, es decir, tomando $(z^*, \lambda^*)$.

%Definición de los puntos estacionarios del MPEC
% Flegel and Kanzow 2003
En el contexto de los MPEC en \cite{Flegel2003AFJ} se exponen varios tipos de puntos estacionarios que son cruciales para analizar la optimalidad, los cuales son los siguientes:
\begin{definition}[Punto Factible]
    Un \textit{punto factible} $z$ del MPEC se llama débilmente estacionario si existe un multiplicador de Lagrange $\lambda = (\lambda^g, \lambda^h, \lambda^G, \lambda^H)$ tal que se cumplen las siguientes condiciones:
    
%TODO: USar multiplicadores analogos a como esta en el bijnivel de miu lambda beta 
% EN este orde  mui-g, alpha-h beta-G gamma-H
\begin{align}
& \nabla f(z) + \sum_{i=1}^m \lambda_i^g\nabla g_i(z) + \sum_{i=1}^p \lambda_i^h\nabla h_i(z) - \sum_{i=1}^{\ell} [\lambda_i^G\nabla G_i(z) + \lambda_i^H\nabla H_i(z)] = \vec{0} \notag\\
    & \quad \begin{aligned}
        & \lambda_\alpha^G \text{ libre}, \quad \lambda_\beta^G \text{ libre}, \quad \lambda_\gamma^G = 0, \\
        & \lambda_\gamma^H \text{ libre}, \quad \lambda_\beta^H \text{ libre}, \quad \lambda_\alpha^H = 0, \\
        & g(z) \leq 0, \quad \lambda^g \geq 0, \quad (\lambda^g)^T g(z) = 0.
    \end{aligned} \\
& \label{eq:Definicion_punto_debilemente_estacionario} \notag
\end{align}
\end{definition}

% Definición de puntos estacionarios
Este concepto de estacionariedad, sin embargo, es una condición relativamente débil. Existen conceptos más fuertes de estacionariedad que se derivan y estudian en otros lugares. En particular, se tienen las siguientes definiciones:
% un punto factible $z$ con el correspondiente multiplicador de Lagrange $\lambda = (\lambda^g, \lambda^h, \lambda^G, \lambda^H)$ se llama:

\begin{itemize}
% C-estacionario
\item \begin{definition}[Punto C-estacionario]
  El punto $z$ es: \textit{C-estacionario} si, para cada $i \in \beta$, $\lambda_i^G\lambda_i^H \geq 0$ se cumple.
\end{definition}
%M-Estacionario
\item \begin{definition}[Punto M-estacionario]
    El punto $z$ es: \textit{M-estacionario} si, para cada $i \in \beta$, o bien $\lambda_i^G,\lambda_i^H > 0$ $\vee$ $\lambda_i^G \lambda_i^H = 0$.
\end{definition}
%Fuertemente estacionario
\item \begin{definition}[Punto Fuertemente estacionario]
    El punto $z$ es: \textit{fuertemente estacionario} o \textit{estacionario primal-dual} si, para cada $i \in \beta$, $\lambda_i^G, \lambda_i^H \geq 0$.
\end{definition}
% A-estacionario
%\item \begin{definition}[Punto A-estacionario]
%   Sea $z^*$ un punto débilmente estacionario del MPEC \eqref{eq:DefMpec} será \textit{A-estacionario} si existe un multiplicador de Lagrange correspondiente $\lambda^*$ tal que:
%\begin{equation}
%(\lambda_i^G)^* \geq 0 \quad \vee \quad (\lambda_i^H)^* \geq 0 \quad \forall i \in \beta \notag
%\end{equation}
%\end{definition}
\end{itemize}

%TODO:
Estas definiciones son condiciones necesarias de optimalidad bajo ciertas condiciones de regularidad, las m;as fuerte basda en la MPEC-LICQ conlleva al siguiente resultado:
% Teorema que si z^* es min local del MPEC Si se cumple la LICQ entonces existe .....
\begin{theorem} 
Sea $z^* \in \mathbb{R}^n$ un mínimo local del MPEC \eqref{eq:DefMpec}. Si MPEC-LICQ se cumple en $z^*$, entonces existe un único multiplicador de Lagrange $\lambda^*$ tal que $(z^*, \lambda^*)$ es \textit{fuertemente estacionario}.
\end{theorem}

% Modelación en Julia
% Se explica de que va BilevelJump
\section{Modelación en Julia}
% Que es BilevelJump
BilevelJuMP.jl es un paquete de Julia diseñado para modelar y resolver problemas de \textbf{optimización binivel}, también conocidos como problemas de optimización de dos niveles o jerárquica, para más detalles ver \cite{BilevelJump}.
Estos problemas se caracterizan por tener dos niveles de decisión: un nivel superior y un nivel inferior, donde las decisiones del nivel superior influyen en las decisiones del nivel inferior, y viceversa \cite{BilevelJump}.
% Que problemas resuelve
Este paquete permite abordar una amplia variedad de tipos de problemas. En este trabajo usaremos las siguientes bibliotecas:
% Añadir explicación MPEC
\subsection*{JuMP}
% TODO: PONEWr esto  Permite gestionar una variedad de restricciones compatibles con JuMP, tales como restricciones cuadráticas, no lineales y enteras.
%Facilita la transformación de problemas binivel en problemas de programación matemática con restricciones de equilibrio (MPEC) y ofrece métodos para abordar las restricciones de complementariedad que surgen en estos casos.
  %TOD: Poner esto despues de expluicar las reformulaciones
\subsection*{BilevelJuMP}
BilevelJuMP.jl facilita la modelación de problemas que pueden representarse en la sintaxis de JuMP, ver documentación en \cite{JuMPPaper}, incluyendo restricciones lineales y no lineales, variables continuas y enteras, y diferentes tipos de objetivos.
\begin{itemize}
    \item \textbf{Problemas de optimización binivel generales:} BilevelJuMP.jl facilita la modelación de problemas que pueden representarse en la sintaxis de JuMP, ver documentación en \cite{JuMPPaper}, incluyendo restricciones lineales y no lineales, variables continuas y enteras, y diferentes tipos de objetivos.
    
    \item \textbf{Problemas con restricciones de determinados tipos en el nivel superior:} Permite gestionar una variedad de restricciones compatibles con JuMP, tales como restricciones cuadráticas, no lineales y enteras.
    
%TODO: MOdificar esto para decir que hace esto.
    
    
    \item \textbf{Problemas con diferentes tipos de reformulaciones:} Los usuarios pueden experimentar con diversas reformulaciones para las restricciones de complementariedad en los problemas MPEC, incluyendo SOS1, restricciones de indicador, Fortuny-Amat y McCarl (Big-M), entre otros.
    %TODO: POner que es comun a los dos 
    \item \textbf{Problemas que requieren solvers MIP y NLP:} BilevelJuMP.jl puede utilizar tanto solucionadores de \textbf{programación lineal mixta entera (MIP)} como solucionadores de \textbf{programación no lineal (NLP)}, dependiendo de las características del problema y la reformulación elegida.
\end{itemize}

%\subsection{Limitaciones del BilevelJuMP.jl}

A pesar de sus capacidades, BilevelJuMP.jl presenta algunas limitaciones, entre las cuales se encuentran:

\begin{itemize}
    \item Enfrentar dificultades en problemas altamente no lineales o con estructuras de optimización complejas que no se puedan representar adecuadamente en la sintaxis de JuMP.
    \item Existencia de ciertas restricciones que podrían no ser compatibles o que requieren transformaciones adicionales que podrían complicar el modelo.
    \item El problema es de gran escala afectando el rendimiento del solver, el cual puede ser un factor crítico.
    \item La formulación y resolución de problemas muy específicos o especializados podrían no estar completamente optimizadas en el paquete.
\end{itemize}


% Explicación algoritmos 
\section{Métodos de Reformulación para Optimización Binivel}
La optimización binivel presenta desafíos particulares debido a su naturaleza jerárquica y las condiciones de complementariedad resultantes. A continuación, se presentan los principales métodos de reformulación implementados en la literatura, basados en la transformación del MPEC \eqref{eq:KKT_Optimista}.
\subsection{Método Big-M}

El método Big-M (Fortuny-Amat y McCarl)  es una técnica fundamental para reformular problemas de optimización binivel en problemas MPEC. 
Este método aborda específicamente las condiciones de complementariedad que surgen en estas reformulaciones, transformando el problema original en un problema de programación lineal mixta entera (MILP).

La reformulación mediante Big-M introduce un parámetro M suficientemente grande y variables binarias para transformar las condiciones de complementariedad no lineales en restricciones lineales. Para una condición de complementariedad 
$v_j(x,y)\lambda_j = 0$ en \eqref{eq:KKT_Optimista}, el método introduce una variable binaria  $f_j \in \{0,1\}$ y cotas superiores $M_p$, $M_d$ bajo  las siguientes restricciones:

\begin{align}
    v_j(x,y) &\geq -M_p(1 - f_j) \notag \\
    \lambda_j &\leq M_d f_j \\
	%TODO: EN vez sde f_j poner \delta_j
    f_j &\in \{0,1\} \notag
\end{align}

donde $M_p$ y $M_d$ son valores grandes para las variables primales y duales, respectivamente. La efectividad del método depende crucialmente de la selección apropiada de estos valores, que deben ser suficientemente grandes para no excluir la solución óptima, pero no excesivamente grandes para evitar inestabilidades numéricas \cite{BilevelJump}.


\subsection{Método SOS1}

El método de Conjuntos Ordenados Especiales tipo 1 (SOS1) evita el uso de parámetros Big-M mediante restricciones de tipo conjunto. Para cada par complementario $(v_j, \lambda_j)$:

\begin{equation}
%DONDe SOS1 es v_j * \lambda_j 
[v_j(x,y); \lambda_j] \in \text{SOS1} \label{eq:SOS1_reform}
\end{equation}

Esta restricción fuerza que al menos una variable en el par sea cero, preservando la no linealidad original sin necesidad de cotas. Es particularmente eficaz en problemas con estructura cónica \cite{BilevelJump}.


\subsection{Método ProductMode}

El método ProductMode representa un enfoque directo para manejar las condiciones de complementariedad en su forma de producto original. Este método es particularmente útil cuando se trabaja con solucionadores de programación no lineal (NLP), aunque no garantiza la optimalidad global.

La implementación del ProductMode mantiene la restricción de complementariedad en su forma original:

\begin{equation}
    v_j(x,y) \cdot \lambda_j \leq t \label{eq:ProductMode_reg}
\end{equation}

donde $t > 0$ es un parámetro de regularización pequeño. Esta formulación, aunque no satisface las condiciones de calificación de restricciones estándar, es útil para obtener soluciones iniciales y puede ser especialmente efectiva cuando se combina con solucionadores NLP, ver \cite{BilevelJump}.

% TODO: hjacer suempre cierre capitulo, y dicinedo que se dijo y motivar

%\subsection{Comparación de Métodos}
%La Tabla \ref{tab:comparacion_metodos} resume los requisitos y características de cada método:
%
%\begin{table}[H]
%\centering
%\begin{tabular}{l|l|l}
%\textbf{Método} & \textbf{Solver Requerido} & \textbf{Ventajas} \\ \hline
%Big-M & MIP & Estabilidad numérica controlada \\
%SOS1 & MIP con SOS1 & Sin parámetros ad-hoc \\
%ProductMode & NLP & Manejo de no linealidades \\
%\end{tabular}
%\caption{Comparación de métodos de reformulación}
%\label{tab:comparacion_metodos}
%\end{table}
%
%Para casos con restricciones no lineales $v_j(x,y)$, se recomienda combinar ProductMode con técnicas de linealización por tramos \cite[Apéndice B]{BilevelJump}. Todos estos métodos están implementados en BilevelJuMP.jl, permitiendo experimentación con múltiples reformulaciones \cite[Sección 4]{BilevelJump}.
%
%\section{Add}
%\begin{equation}
%    \min_{x,y,\lambda_j,f_j} F(x,y)
%\end{equation}
%
%sujeto a:
%\begin{align}
%    & g_i(x,y) \leq 0, && i = 1,\ldots,q, \\
%    & \nabla_y f(x,y) + \sum_{j=1}^s \nabla_y v_j(x,y) \lambda_j = 0, \\
%    & v_j(x,y) \leq 0, && j = 1,\ldots,s, \\
%    & \lambda_j \geq 0, && j = 1,\ldots,s, \\
%    & v_j(x,y) \geq -M_p \cdot (1-f_j), && \text{(no lineal si } v_j \text{ es no lineal)} \\
%    & \lambda_j \leq M_d \cdot f_j, && j = 1,\ldots,s, \\
%    & f_j \in \{0,1\}, && j = 1,\ldots,s.
%\end{align}
%\include{MainMatter/Proposal}
\chapter{Detalles de Implementación y Experimentos}\label{chapter:implementation}

%region Nomenclatura 
En este capítulo se mostrará la forma y los pasos para generar el problema deseado en el contexto de un generador de puntos estacionarios para problemas binivel. Este capítulo aborda el diseño y la implementación del generador, explicando detalladamente las metodologías empleadas, los criterios necesarios para asegurar la validez de los puntos obtenidos y los algoritmos utilizados en cada etapa del proceso.

\section{Nomenclatura a utilizar:}
%TODO: Recorda el problema y la nomenclatura en el anexo 
% Recordar que el problema va hacia el kkt 
% Se busca un punto estacionario

\begin{table}[H]
    \centering
    \caption{Nomenclatura a utilizar}
    \begin{tabular}{l m{360pt}}
        $F(x,y)$              & Función del líder.                                                                                                          \\
        $ g_i(x,y) $              & Restricciones de desigualdad del líder   $ i=1\ldots q$.                                                                                                        \\
        $ h_i(x,y) $                 & Restricciones de igualdad del líder   $ i=1\ldots r$.                    \\
        $ f(x,y) $           & Función del seguidor.                                                               \\
        $ v_j(x,y) $              &  Restricciones de desigualdad del seguidor $j=1\ldots s$.   \\
        $ u_j(x,y) $     & restricciones de igualdad del seguidor $j=1\ldots t$. \\
        $ v_{j}^{\star} $    & Restricción de desigualdad del seguidor activa después de modificarse el problema   $j=1\ldots s$.          \\
        $z_{est}$         & Punto que se desea que sea estacionario.\\
        $J_0^g$             & Índices activos del líder. \\
        $J_0^v$                & Índices activos del seguidor. \\
        $ \alpha  $             & Vector de entrada de dimensión igual cantidad variables seguidor.                                                                                                      \\
        $\mu_i $                & Multiplicador asociado al $g_i(x,y)$ en el líder.  \\
        $ \beta_j $               & Multiplicador asociado al $v_{j}^{\star}$ en el líder.          \\
        $ \lambda_j $              & Multiplicador asociado al $v_{j}^{\star}$ en el seguidor.\\
        $\gamma_j$                & Valor asociado a cada $v_{j}^{\star}$ con $\lambda_j=0$.\\
    \end{tabular}

    \end{table}

\newpage
%endregion
\section{Generación del problema}
A continuación se mostrarán los pasos a seguir para generar el problema deseado donde el punto sea estacionario de la clase específica.

\subsection{Requerimientos de entrada}

Para generar un problema es necesario inicialmente introducir un problema de optimización binivel del tipo SLSF, un punto ($z_{est}$), el cual
se desea que sea estacionario y los multiplicadores con signo o valores con ciertas condiciones en dependencia del tipo de estacionariedad requerida.

%
El problema de entrada tiene que cumplir con ser un programa de optimización binivel como \ref{eq:DefBinivelOptimista}. 
Debe introducirse también sus índices activos de las restricciones de desigualdad en cada nivel para $z_{est}$ :
\begin{itemize}
    % Leader
    \item Nivel Superior:
            \begin{equation}
             J_0^G=\{i | g_i(x,y)=0\}
            \label{J_0_level_superior} %al conjunto restricciones de desigualdad que sean índices activos del nivel superior
            \end{equation}\\
    donde estos índices activos tienen un $\mu_i$ relacionados.
        
    %Follower
    \item Nivel Inferior:
                Se tiene en cuenta con respecto a los valores del multiplicador $\lambda_i$ y $v_j(x,y)$.
                
                \begin{itemize}
                    %J_0_v_l_0
                    \item \begin{equation}
                        J_1^v=\{j | v_j(x,y)=0 \land \lambda_j=0 \} %al conjunto restricciones de desigualdad que sean índices activos
                        \end{equation} 
                    %J_0_v_l_positiva
                    \item \begin{equation}
                        J_2^v=\{j | v_j(x,y)=0 \land \lambda_j>0 \}
                        \label{J_0_lambda_pos_level_inferior}
                    \end{equation}
                    %J_Ne_L0_v
                    \item \begin{equation}
                        J_3^v=\{j | v_j(x,y)< 0 \land \lambda_j=0 \}
                        \label{J_neg_lambda_0_level_inferior}
                    \end{equation}
                \end{itemize}
    Cada índice activo tiene multiplicadores $\beta_j$ y en dependencia del caso $\gamma_j$ asociados.
\end{itemize}
%

% Condiciones
%Hablar sobre conocer vector \alpha
Para cada $i \in J_0^G$ (\ref{J_0_level_superior}) debe de asignarse su multiplicador $\mu_i \geq 0$ correspondiente.
Además se debe conocer el valor del $\vec{\alpha}$,
para el nivel inferior debe tenerse en cuenta si $\alpha=\vec{0}$ en cuyo caso afirmativo $\gamma_j$ es un valor de entrada

%
% COndiciones \Beta_i \Gamma_i para puntos estacionarios
Para obtener las diferentes clases de puntos estacionarios debe de introducirse $\beta_j$, $\gamma_j$, en caso de ser este último valor de entrada tal que, los multiplicadores $\beta_j$  y $\gamma_j$ de forma tal que en los indices activos $J_1^v$ (\ref{J_0_lambda_0_level_inferior}): 
\begin{itemize}
    %C estacionario
    %Betai, gammai>0, betai,gammai<0 o 
    %Betai libre gammai=0
    %Gammai libre , betai=0
    \item \textbf{Punto C-Estacionario:}\\
    \begin{equation}
        \beta_j * \gamma_j >=0
        \label{Requisitos puntos C-Estacionario}
    \end{equation}
    %M estacionario
    %Betai, gammai>0,
    %Betai libre gammai=0
    %Gammai libre , betai=0
    \item \textbf{Punto M-Estacionario}\\
    \begin{equation}
        \begin{aligned}
            &\beta_j \land \gamma_j>0\\
            &\beta_j \quad \text{Libre} \quad \gamma_j=0\\
            &\beta_j=0 \quad \gamma_j \quad \text{Libre}\\
        \end{aligned}
        \label{Requisitos puntos M-Estacionarios}
    \end{equation}
    %Fuertemente estacionario
    %Betai, gammai>0,
    \item \textbf{Punto Fuertemente Estacionario}\\
     \begin{equation}
        \beta_j \land \gamma_j \geq 0
    \label{Requisitos puntos Fuertemente Estacionarios}
    \end{equation}    
\end{itemize}
% Especificar que en caso de  que alpha no se el vector nulo se asume que gamma_i es 0
\textbf{En caso que $\vec{\alpha} \neq \vec{0}$ se asume que $\gamma_j=0$ en las condiciones anteriores.}

% Explicar que se hace primero el punto factible
% Subseccion donde se explica que el punto se hace factible inicialmente

\subsection*{Modificación del Problema Original}
Después de tener los datos de entrada necesarios se procede a la modificación
del problema de entrada para que este sea estacionario del tipo requerido, $z_{est}$.

%
% Decir que en caso de que \alpha != de nulo entonces se calcula b_j
Primero en caso de que $\alpha \neq 0$ en los $v_j(x,y) \in J_0^v$ se procede a calcular el $\vec{b_j}$ de la siguiente forma:

%Como calcular b_j
\begin{itemize}
    \item \textbf{ $v_j(x,y) \in J_2^v$ \ref{J_0_lambda_pos_level_inferior}}:
        \begin{equation}
            \vec{b_j}=  \frac{{\alpha} \cdot (-\nabla_{y}{v_j(z_{est})}^T \cdot \alpha)}{\|\mathbf{\alpha} \|_2^2}
        \end{equation}
    \item \textbf{$v_j(x,y) \in J_1^v \lor J_3^v$ (\ref{J_0_lambda_0_level_inferior},\ref{J_neg_lambda_0_level_inferior})}\\
    \begin{equation}
        \vec{b_j}=  \frac{{\alpha} \cdot ((-\nabla_{y}{v_j(z_{est})}^T \cdot \alpha)+\gamma_j)}{\|\mathbf{\alpha} \|_2^2}
    \end{equation}
\end{itemize}


Después de realizado el calculo del $b_j$ se procede a modificar su índice activo correspondiente:

\begin{equation}
	v_{j}^{\star}(z_{est})=v_{j}(z_{est})+ ({\vec{b_j}}^T)\cdot (y_1,y_2,\dots,y_m)
\end{equation}
%

% Explicar que se hace el punto factible en las restricciones
Luego en todas las restricciones del nivel inferior y superior
se evalúa el punto y se verifica si es factible. En caso de no ser factible 
bajo las nuevas restricciones se añaden constantes $c_i$ en el caso de las restricciones del
nivel superior y $c_j$ para las del inferior. Esto conlleva a que sea un punto factible en el conjunto
de restricciones de ambos niveles sin perjudicar las propiedades relacionadas con la convexidad al ser una adicción
de funciones lineales. 

% Explicar que se realiza el KKT del nivel inferior
Al hacer factible se procede a realizar el KKT del nivel inferior con las modificaciones anteriores evaluado en $z_{est}$.

Se plantea las condiciones KKT y se halla $\vec{bf}$ :

\begin{equation}
    \begin{aligned}
        &\nabla_{y}f(x,y)+\sum_{j \in J_2^v}(\lambda_j\nabla_{y}v_{j}^{\star}(x,y))+\vec{bf}=\vec{0}\\
        &\text{KKT del problema del nivel inferior}
    \end{aligned}
    \label{KKT_nivel_inferior}
\end{equation}
\linebreak
%Explicar que como se había hallado ya el KKT del level inf ahora se puede realizar completo
Finalmente podemos construir la reformulación en MPEC como \ref{eq:KKT_Optimista} calculando el $\vec{BF}$ y 
obteniendo el problema de forma reformulada.
% MPEC all KKT 
\begin{table}[H]
    \begin{equation}
        \label{KKT_del_MPEC}
    \end{equation}
	$\nabla_{xy}F(x,y)+\sum_{j\in J_o^g|}(\mu_i\nabla_{xy}g(x,y))+[\nabla_{x,y}\nabla_{y}f(x,y)+\sum_{j \in J_1^vg}\lambda_j\nabla_{xy}\nabla_{y}v_{j}^{\star}(x,y)]\alpha+\sum_{j \in J_1^v \cup J_2^v|}(\beta_j\nabla_{xy}v_{j}^{\star}(x,y))+\vec{BF}=\vec{0}$  \label{KKT_del_MPEC}
\caption*{KKT del MPEC}
\end{table}

% Explicación algorítmica y manual de usuario

\section{Implementación Algorítmica y Guía de Usuario}
%region Introducción de la sección
En la implementación computacional del generador propuesto en la sección anteriores
se utilizó el lenguaje de programación \textbf{Julia}, ver \cite{Juliadocs}, dado a su versatilidad
en expresiones y funciones matemáticas implementadas en su paquete base y sus bibliotecas externas como 
\textbf{BilevelJuMP}, ver \cite{BilevelJump}, que permite resolver problemas binivel SLSF lineales y cuadráticos
sin tener que transformar el problema original y \textbf{JuMP} \cite{JuMPPaper} el cual, es una interfaz robusta para la optimización general,
todo ello con unas excelentes prestaciones de cómputo. Por ello se brindará una guía de usuario para el uso del generador.

%\subsection{Guía de Usuario}
% Después compilar a ProblemGenerator
La implementación algorítmica se ha llevado a cabo mediante la creación de una biblioteca de Julia
llamada \textbf{ProblemGenerator}, con una sintaxis \textit{JuMP-like} intuitiva.
Primero debe tenerse un problema de optimización Binivel SLSF planteado como el \ref{eq:DefBinivelOptimista}, el punto que se desea que sea estacionario ($z_{est}$), los índices activos descritos anteriormente 
según el caso así como sus multiplicadores correspondientes, que cumplirán las propiedades del tipo de estacionariedad requerida y el $\vec{\alpha}$.
%endregion

%\subsection{Ejemplo a utilizar}
Para una mayor facilidad de compresión del uso de la biblioteca se utiliza este problema ejemplo:
\begin{equation}
    \begin{aligned}
        & \underset{x_1, x_2}{\text{mín}} 
        && x_1^2 y_1^2 y_2 + x_2 \\
        & \text{s.t.} 
        && x_1 + y_2 - y_1 \leq 9, \\
        & 
        && \underset{y_1, y_2}{\text{mín}} 
        \quad x_2^2 y_1^2 y_2 + x_1 \\
        & 
        && \text{s.t.} 
        \quad x_1^2 y_1^2 + x_2 \leq 0.
    \end{aligned}
    \label{ProblemaEjemplo}
\end{equation}
%Dependencias para la biblioteca
%\subsection{Dependencias}
Se debe tener instalado en el entorno de desarrollo de Julia los siguientes módulos:
%Se debe tener en cuenta que este paquete instalará las siguientes dependencias
\begin{itemize}
    \item Symbolics 
    \item LinearAlgebra
\end{itemize}

% Importar la Biblioteca
%\subsection{Importación de la biblioteca}
Para la utilización de la biblioteca se procede a su importación de la siguiente forma:
\begin{lstlisting}[caption=Importar el Módulo]
    using ProblemGenerator
\end{lstlisting}

% Crear modelo base
%\subsection{Crear el modelo base}
Para comenzar debe llamarse a la función \textbf{GeneratorModel} para crear el modelo base
inicial.

\subsubsection{Para $\vec{\alpha}=\vec{0}$ }

%Crear para alpha nulo
Para crear un modelo donde $\vec{\alpha}=\vec{0}$:
   
        \begin{lstlisting}[caption={Crear el modelo para $\vec{\alpha}=\vec{0}$}]
            # Crear el modelo base del generador alpha=0
            model=GeneratorModel()
        \end{lstlisting}
        
    
  
%Crear para alpha no nulo
%
\subsubsection{Para $\vec{\alpha} \neq \vec{0}$ }

Para crear un modelo donde $\vec{\alpha}\neq \vec{0}$,
se tiene que pasar como parámetro el valor del $\alpha$,
el cual será un vector de \textit{Number}:

\begin{lstlisting}[caption={Crear el modelo para $\vec{\alpha } \neq \vec{0}$}]
    # Crear el modelo base del generador alpha!=0
    alpha_vec=Vector::{Number}
    model=GeneratorModel(alpha_vec)
\end{lstlisting}
%



% Explicar las funciones Upper y Lower
%\subsection{Función para diferenciar los niveles}
En caso que se requiera como entrada un \textbf{Problem}, se debe declarar
de que nivel se trata de evaluar el \textbf{model} en las funciones \textbf{Upper} para el nivel 
superior y \textbf{Lower} para el inferior.

\subsubsection{Para referirse al nivel superior}
\begin{lstlisting}[caption={Referirse al nivel superior}]
    # Referirse nivel superior
    Upper(model)
\end{lstlisting}
\subsubsection{Para referirse al nivel inferior}
\begin{lstlisting}[caption={Referirse al nivel inferior}]
    # Referirse nivel inferior
    Lower(model)
\end{lstlisting}


%\subsection{Declarar Variables}
Las variables deben de 
declararse bajo la siguiente macro \textbf{@myvariables}, la cual 
recibe una función \textbf{Upper} o \textbf{Lower} que recibe el modelo
para las variables del nivel superior y las del nivel inferior respectivamente.

\subsubsection{Introduccion de  las variables del nivel superior }
%Ejemplo de introducir al level superior

\begin{lstlisting}[caption={Introducir las variables del nivel superior}]
    # Se declara en el nivel superior las variables x_1, x_2
    @myvariables Upper(model) x_1, x_2
\end{lstlisting}

\subsubsection{Introducción de las variables del nivel inferior}
%Ejemplo para introducir las variables del nivel inferior

\begin{lstlisting}[caption={Introducir las variables del nivel inferior}]
   # Se declara en el nivel inferior las variables y_1, y_2
   @myvariables Lower(model) y_1,y_2
\end{lstlisting}


%Declarar las funciones objetivo
\subsection{Declarar Funciones Objetivo}

Para declarar las funciones objetivos debe asumirse que en ambos casos es un 
problema de minimización. Se utilizará la función \textbf{SetObjectiveFunction}
que recibe un \textbf{Problem} y una expresión de \textbf{Num}.

% Declaración de Función Objetivo de nivel Superior
\subsubsection{ Declaración de una función objetivo del nivel superior}
Con: $$\min (x_1^2*y_1^2*y_2) + x_2$$
\begin{lstlisting}[caption={ Declarar una función objetivo del nivel superior}]
    # Declarar la funcion objetivo del nivel superior
    # Min de ((x_1^2)*(y_1^2)*(y_2))+x_2
    SetObjectiveFunction(Upper(model),((x_1^2)*(y_1^2)*(y_2))+x_2)
\end{lstlisting}

\subsubsection{Declaración de una función objetivo del nivel inferior}
%Declaración de Función objetivo Nivel inferior

Con: $$\min (x_2^2*y_1^2*y_2)+x_1$$
\begin{lstlisting}[caption={Declarar una función objetivo del nivel inferior.}]
    # Declarar la funcion objetivo del nivel inferior
    # Min de ((x_2^2)*(y_1^2)*(y_2))+x_1
    SetObjectiveFunction(Lower(model),((x_2^2)*(y_1^2)*(y_2))+x_1)
\end{lstlisting}

% Definición de tipo de índices activos.
\subsection{Definición del conjunto de Índices activos}

Antes de ilustrar como introducir las restricciones se va a explicar 
que los tipos de índices activos.

\begin{itemize}
    \item \textbf{Ambos Niveles}:
        \begin{itemize}
            \item \textbf{Normal} si no es un índice activo.
        \end{itemize}
    \item \textbf{Nivel Superior}:
     \begin{itemize}
        \item \textbf{J\_0\_g} Si es un índice como en \ref{J_0_level_superior}.
     \end{itemize}
    \item \textbf{Nivel Inferior}:
    \begin{itemize}
        \item \textbf{J\_0\_LP\_v} Si es un índice como en \ref{J_0_lambda_pos_level_inferior}.
        \item  \textbf{J\_0\_L0\_v} Si es un índice como en \ref{J_0_lambda_0_level_inferior}.
        \item  \textbf{J\_Ne\_L0\_v} Si es un índice como en \ref{J_neg_lambda_0_level_inferior}.
    \end{itemize}
\end{itemize}


%Ejemplo de como se debe de expresar en Julia
Los \textbf{RestrictionSetType} deben expresar en código de esta forma:
\begin{lstlisting}[caption={Definir el conjunto de índice activo}]
    Normal 
    J_0_g 
    J_0_LP_v
    J_0_L0_v
    J_Ne_L0_v
\end{lstlisting}

Los \textbf{RestrictionSetType} son un \textit{enum} de Julia.

% Declaración de restricciones
%\subsection{Declarar Restricciones}

Para declarar las restricciones se brindan dos funciones:
\begin{itemize}
    \item \textbf{SetLeaderRestriction:} Para el nivel superior.
    \item \textbf{SetFollowerRestriction:} Para el nivel inferior.
\end{itemize}
\begin{itemize}
    % Declarar sobre el nivel superior
    \item \textbf{Nivel Superior:}\\

% Declaración para el nivel Superior
Para declarar las restricciones del nivel superior debe por cada restricción
llamarse a la función \textbf{SetLeaderRestriction} con:
\begin{itemize}
    \item El modelo (\textbf{model})
    \item La expresión de la restricción, del tipo \textbf{Num}
    \item El tipo de restricción, del tipo \textbf{RestrictionSetType}
    \item El valor de $\mu_i$ correspondiente, del tipo \textbf{Number} 
\end{itemize}

Ejemplo para introducir la restricción:
\begin{align*}
    &x_1+y_2-y_1 \leq 9 \\
    &\text{Índice activo del tipo J\_0\_g \ref{J_0_level_superior}}\\
    & \mu_i=0.3
\end{align*}

\begin{lstlisting}[caption={Introducir restricción del nivel superior}]
    # Ejemplo de restriccion del nivel superior
    SetLeaderRestriction(model,x_1+y_2-y_1<=9,J_0_g,0.3)
\end{lstlisting}


    % Declarar sobre el nivel inferior
\item \textbf{Nivel Inferior:}\\
Para declarar las restricciones del nivel inferior debe por cada restricción
llamarse a la función \textbf{SetFollowerRestriction} con:
\begin{itemize}
   \item El modelo (\textbf{model})
   \item La expresión de la restricción, del tipo \textbf{Num}
   \item El tipo de restricción, del tipo \textbf{RestrictionSetType}
   \item El valor de $\beta_j$ correspondiente, del tipo \textbf{Number}
   \item El valor de $\lambda_j$ correspondiente, del tipo \textbf{Number}
   \item El valor de $\gamma_j$ correspondiente en caso de ser valor de entrada, del tipo \textbf{Number}
\end{itemize}


Ejemplo para introducir la restricción:

\begin{itemize}
    
    \item \textbf{$ \lambda_j \neq 0$:}\\
    \begin{align*}
        &((x_1^2)*(y_1^2))+x_2 \leq 0\\
        &\text{Índice activo del tipo J\_Ne\_L0\_v} \ref{J_neg_lambda_0_level_inferior}\\
        & \beta_j=0.1\\
        & \lambda_j=0.5 \\
    \end{align*}
    \begin{lstlisting}
        # Para caso lamda_j=0
        SetFollowerRestriction(model,((x_1^2)*(y_1^2))+x_2<=0,J_Ne_L0_v,0.1,0.5,0)
    \end{lstlisting}
    

    
   
    \item \text{$\lambda_j= 0$:}\\
    
    Análogo al caso anterior pero con $\lambda_j=0$, por lo que $\gamma_j$ valor de entrada
    \begin{equation*}
        \beta_j=0.1, \quad \lambda_j=0, \quad \gamma_j=0.4
    \end{equation*}
    \begin{lstlisting}
        # Para caso lamda_j!=0
        SetFollowerRestriction(model,((x_1^2)*(y_1^2))+x_2<=0,J_Ne_L0_v,0.1,0,0.4)
    \end{lstlisting}

\end{itemize}


% Finalizar el itemize que tiene de explicación como introducir las restricciones de ambos niveles
\end{itemize}


\subsection{Introducir el Punto}
Ahora debe de introducirse el valor del punto que debe ser estacionario de la clase seleccionada.
Debe de tenerse en cuenta que para todas las variables declaradas en ambos niveles debe de definirse el valor de la componente.

\begin{lstlisting}[caption={Introducir el punto $(1,1,1,1)$}]
    # Introducir el punto (1,1,1,1)
    SetPoint(model,Dict(x_1=>1,x_2=>1,y_1=>1,y_2=>1))
\end{lstlisting}

\subsection{Generar el Problema}
Finalmente al tener todos los datos previos introducidos se llama al \textbf{CreateProblem} pasándole como parámetro el \textit{model} 
y generará dicho problema imprimiendo en consola este.

\begin{lstlisting}[caption={Generar el problema}]
    # Llamar para generar el problema
    problem=CreateProblem(model)
\end{lstlisting}

    
\subsection{Ejemplo completo}
Se muestra el ejemplo completo para el problema antes propuesto:
\begin{lstlisting}[caption={Script}]
    # Importar dependencias necesarias
    using ProblemGenerator
    # Crear el modelo base del generador alpha=0
    model=GeneratorModel()
    #Declaracion de variables
    # Se declara en el nivel superior las variables x_1, x_2
    @myvariables Upper(model) x_1, x_2
    # Se declara en el nivel inferior las variables y_1, y_2
    @myvariables Lower(model) y_1,y_2
    # Declarar la funcion objetivo del nivel superior
    # Min de ((x_1^2)*(y_1^2)*(y_2))+x_2
    SetObjectiveFunction(Upper(model),((x_1^2)*(y_1^2)*(y_2))+x_2)
    # Ejemplo de restriccion del nivel superior
    SetLeaderRestriction(model,x_1+y_2-y_1>9,J_0_g,0.3)
    # Declarar la funcion objetivo del nivel inferior
    # Min de ((x_2^2)*(y_1^2)*(y_2))+x_1
    SetObjectiveFunction(Lower(model),((x_2^2)*(y_1^2)*(y_2))+x_1)
    # Para caso lamda_j!=0
    SetFollowerRestriction(model,((x_1^2)*(y_1^2))+x_2<=0,J_Ne_L0_v,0.1,0,0.4)
    # Introducir el punto (1,1,1,1)
    SetPoint(model,Dict(x_1=>1,x_2=>1,y_1=>1,y_2=>1))
    # Llamar para generar el problema
    problem=CreateProblem(model)
\end{lstlisting}


\subsection{Documentación Oficial}
Para mayor información visitar la documentación oficial del generador:

\begin{center}
\href{https://fvsb.github.io/Tesis/}{https://fvsb.github.io/Tesis/}. 
\end{center}


\chapter{Experimentación}
En este capítulo se experimentará tomando una serie de problemas, ver \cite{BolibTestProblems}, 
que sean: Lineales, Cuadráticos y No Convexos. En ella primeramente utilizaremos las bibliotecas de Julia
para obtener puntos que sean mínimos locales, después añadir valores aleatorios a esos puntos y posteriormente 
generar problemas estacionarios del tipo: fuerte, M y C . Posteriormente estos problemas modificados serán nuevamente
ejecutados por los algoritmos tradicionales de Julia para conocer su efectividad con respecto al valor de la función objetivo del nivel superior.


Para ello tomaremos 15 problemas de optimización binivel.
Entre ellos habrá 3 clases: Lineales, Cuadráticos y No Convexos, donde habrá 5 problemas de cada clase.
Estos han sido extraídos de \cite{Floudas1999HandbookOT} para las dos primeras clasificaciones y \cite{BolibTestProblems} para la última.

\newpage
%\section{Problemas Escogidos}
Los problemas escogidos para la experimentación fueron:

%Tabla con los problemas escogidos
\begin{table}[h!]
\centering
\caption{Problemas Seleccionados}
\begin{tabular}{ | m{5cm} | m{5cm} | m{5cm} | }
  
  \hline
  \textbf{No Convexos} & \textbf{Lineales} & \textbf{Cuadráticos} \\
  \hline
  MitsosBarton2006Ex312 & ex9.1.1 & ex9.2.1 \\
  \hline
  MitsosBarton2006Ex313 & ex9.1.2 & ex9.2.2 \\
  \hline
  MitsosBarton2006Ex314 & ex9.1.8 & ex9.2.3\\
  \hline
  MitsosBarton2006Ex323 & ex9.1.9 & ex9.2.4\\
  \hline
  MorganPatrone2006a & ex9.1.10 & ex9.2.5 \\
  \hline
\end{tabular}
\end{table}



\newpage
\section{Modelación de la experimentación}
Se describirá el proceso de generar la experimentación. 
Todos los valores, en esta parte, han sido redondeados por exceso a dos cifras después de la coma. 

%\subsubsection{Obtención de los óptimos}
Inicialmente se necesita obtener óptimos de los problemas con los paquetes convencionales de Julia
cuya forma de resolver dependerá de la clase del problema. A continuación definiremos este tipo de clases y como se obtienen los óptimos. 

% Definor problemas lioneales

\begin{itemize}
    \item \textbf{Problemas Lineales y Cuadráticos :}\\
            Un problema binivel lineal es aquel el cual los problemas del nivel superior e inferior son lineales, análogamente se define el problema cuadrático.
            % Escribir la definición matemática
            
            Con dicho problema se introduce los datos en la interfaz de \textbf{BilevelJuMP}, ver \cite{BilevelJump}, con el cual se utilizan 3 técnicas entre las ofrecidas por esta:
            \begin{itemize}
                \item \textbf{Big-M :} Con el optimizador High-Performance Solver for Linear Programming (HiGHS) y los valores $\text{primal big M} = 100, \quad \text{dual big M} = 100$.
                \item \textbf{SOS1 :} Con el optimizador Solving Constraint Integer Programs (SCIP).
                \item \textbf{ProductMode :} Con el optimizador Interior Point Optimizer (Ipopt).
            \end{itemize} 
            Cada uno de los resultados de evaluar el problema en cada forma anterior se guarda en un formato \textit{.xlsx}
            donde por cada optimizador se guarda los parámetros:
            \begin{itemize}
                \item Estatus del Primal, el cual define si es un punto factible o no.
                \item Estatus de la Finalización, si terminó porque encontró un óptimo o se estancó en un óptimo local.
                \item Valor de la función objetivo del nivel superior.
                \item El punto óptimo encontrado, en caso de ser hallado.
            \end{itemize} 
            Posteriormente se analizan los resultados de dichos métodos, se selecciona el de mejor evaluación de la función objetivo.
    \item \textbf{Problemas No Convexos :}\\
            Al \textbf{BilevelJuMP} no contar con soporte para esta clase de problemas se utiliza \textbf{JuMP}, ver \cite{JuMPPaper}.
            Por ello se utiliza la reformulación KKT como la de \ref{eq:KKT_Optimista} y se procede a utilizar la interfaz brindada por este. Para el caso de las restricciones de 
            complementariedad se utiliza \textbf{Complementarity}, ver \cite{Complementarityjl}, con el optimizador Ipopt y análogo al caso anterior se extraen los mismos datos.
\end{itemize}


\subsubsection{Generación de los problemas}
% Como se modifica el punto
Se toma el problema original de entrada y el punto obtenido en el paso anterior el cual en cada componente se le hace suma un
valor aleatorio entre $1e-10$ y $5$, siendo este modificado: $z^*_0$. 
% Que se generan 3 clases de problemas estacionarios
Se generan los 3 problemas bajo los 3 tipos de estacionariedad descritos en cada uno, 
% Se toma \alpha=0 y \alpha!=0
además en cada uno se toma la opción de $\vec{\alpha}=\vec{0}$ y $\vec{\alpha}\ne \vec{0}$.
Para los $\vec{\alpha}\neq \vec{0}$ se genera un vector aleatorio donde cada componente está entre $1e-10$ y $3$.
% Como se dividen los indices activos
Luego para los conjuntos de índices activos de las $v_{j}s$ se dividen en $1/2$ del tipo $J_1^v$ (\ref{J_0_lambda_0_level_inferior}) y $1/4$ para los dos restantes.
% Explicar como se modifican los indices activos 
Con respecto a la selección de los multiplicadores $\beta_j$ y $\gamma_j$ se 
se generan valores aleatorios entre $1e-10$ y $10$ en caso que estos no tengan que ser $0$, 
para los casos en que haya más de una combinación de los multiplicadores con respecto a su igualdad a $0$
se toma un valor aleatorio generado por una distribución uniforme discreta.
% Como se guarda
Finalmente cada problema generado es guardado en un archivo \textit{xlsx} con la siguiente designación:
\textit{(nombre del problema)\_(Tipo de punto estacionario)(generator)\_alpha\_((non\_zero) si $\alpha \neq 0$ y (zero) si $\alpha = 0$).xlsx},
donde se guarda: 
\begin{itemize}
    \item Las expresiones de las funciones objetivo de ambos niveles y su valor evaluado en el punto.
    \item Las restricciones de ambos niveles con sus multiplicadores respectivos, el tipo de indice activo y la evaluación de dicha función en el punto.
    \item El punto $z^*_0$.
    \item El vector $\vec{bf}.$
    \item El vector $\vec{BF}$.
    \item El multiplicador $\vec{\alpha}$.
\end{itemize}

\subsubsection{Comparación de los algoritmos de Julia}
Después de tener generados los problemas se utilizan los mismos métodos de Julia mencionados anteriormente
para obtener óptimos de cada problema generado y se elige como representante el que más haya superado su óptimo o en caso de no superar el que mayor distancia tenga con el valor objetivo inicial.

\subsection{Resultados:}
Se presentan los resultados seleccionados bajo los criterios expuestos anteriormente en las siguientes tablas que contienen:
% Explicación de las tablas
\begin{itemize}
    \item El nombre del problema original desde el cual fue modificado para que fuese estacionario de la clase deseada. 
    \item El punto al cual se forzó ser estacionario de la clase requerida y no de un subconjunto de esta, por ejemplo si es C-estacionario no es M-estacionario.
    \item La evaluación de la función objetivo del punto estacionario.
    \item El punto óptimo hallado por los algoritmos de Julia.
    \item La evaluación de la función objetivo del óptimo.
    \item Método seleccionado.
\end{itemize}

\subsubsection{Problemas Lineales} 
% Tabla con los problemas lineales escogidos
%- Lineales
%	- C-Estacionario: ex9.1.8
%	- Fuertemente: ex9.1.10
%	- M-Estacionario:ex9.1.8
%	- Alpha=0: ex9.1.8
\begin{resultstable}{Problemas Lineales Seleccionados}
    \resultrow{C-Estacionario}{ex9.1.8}{(2.55,1.25,4.25,2.15)}{ -1.72}{(3.16,0.79,.5.6,0)}{ -2.72}{Big-M}
    \resultrow{Fuertemente-Estacionario}{ex9.1.10}{(52.15,20.25,104.6,2.05)}{-31.75}{(267,90.35,100,0)}{-393.65}{Big-M}
    \resultrow{M-Estacionario}{ex9.1.8}{(2.55,1.25,4.25,2.15)}{-1.72}{(3.03,2.85,0,0)}{-3.20}{Big-M}
    \resultrow{$\alpha=0$}{ex9.1.8}{(2.55,1.25,4.25,2.15)}{-1.72}{(3.34,0.8,3.68,0)}{-4.05}{Big-M}
    \end{resultstable}

    %\resultrow{$\alpha =0$}{ex9.2.3}{(1.55,2.7,-5.1,-8.65)}{-10.25}{(0,0,-5.1,-10)}{-14.7}{Big-M}


\subsubsection{Problemas Cuadráticos}

%- Cuadráticos
%	- C-Estacionario: ex.9.2.1
%	- Fuertemente: ex9.2.1
%	- M-Estacionario:ex9.2.5 
%	- Alpha=0:ex9.2.3
\begin{resultstable}{Problemas Cuadráticos Seleccionados}
\resultrow{C-Estacionario}{ex9.2.1}{(1.85,4.65)}{116.01}{(4.1,0)}{1.64}{SOS1}
\resultrow{Fuertemente-Estacionario}{ex9.2.1}{(1.85,4.65)}{116.01}{(0.15,1.12)}{33.92}{ProductMode}
\resultrow{M-Estacionario}{ex9.2.5}{(4.75,4.05)}{7.27}{(2.53,2.83)}{0.91}{Product Mode}
\resultrow{$\alpha =0$}{ex9.2.3}{(1.55,2.7,-5.1,-8.65)}{-10.25}{(0,0,-5.1,-10)}{-14.7}{Big-M}
\end{resultstable}

\subsubsection{Problemas No Convexos}

%- No Convexos:
%	- C-Estacionarios: MBEx312
%	- Fuertemente: MBEx314
%	- M-Estacionarios:MBEx312
%	- alpha=0:MBEx313
\begin{resultstable}{Problemas No Convexos Seleccionados}
\resultrow{C-Estacionario}{MitsosBarton2006Ex312}{(0.8,1.85)}{34,94}{(0.8,1.8)}{34.87}{JuMP}
\resultrow{Fuertemente-Estacionario}{MitsosBarton2006Ex314}{(2.1,3.3)}{14.31}{(2.1,-1.45)}{5.52}{JuMP}
\resultrow{M-Estacionario}{MitsosBarton2006Ex312}{(0.8,1.85)}{34,94}{(0.8,-0.89)}{6.48}{JuMP}
\resultrow{$\alpha =0$}{MitsosBarton2006Ex313}{(2.3,4.45)}{-2.15}{(2.3,4.47)}{-2.17}{JuMP}
\end{resultstable}


\subsubsection{Análisis de los resultados}
En el caso de los problemas lineales los 3 algoritmos generalmente dieron con puntos con similar valor de la función objetivo aunque 
se tomó como referencia Big-M dado que fue el más eficiente en tiempo y si redondeamos a 2 números después de la coma son prácticamente todos iguales.
En los problemas cuadráticos se encontró con mayor diferencia entre el valor de la función en el punto estacionario que en el óptimo local alcanzado, en algunos de estos problemas Big-M no pudo hallar un óptimo con calidad posiblemente debido a sus parámetros y continua la tendencia del
problema que parte desde un punto fuertemente estacionario de tener el mayor rango de mejora.
En los no convexos continua esta hipótesis dado que los puntos M-estacionarios y fuertemente estacionarios son los de mayor requerimientos, fueron los que lograron encontrar mejores óptimos. 


\backmatter

\begin{conclusions}
    \chapter{Conclusiones}

    %Explicar que se hizo
    En esta tesis se desarrolló un algoritmo para la generación de problemas de optimización binivel con características específicas de estacionariedad en puntos determinados. El mismo tiene la capacidad de modificar problemas originales para garantizar la factibilidad y estacionariedad en puntos dados, aprovechando las capacidades del lenguaje de programación Julia, que destaca por su alto rendimiento computacional y sintaxis adaptada a problemas de optimización.
    
    % Explicar como fue la experimentación
    La experimentación se llevó a cabo sobre tres categorías fundamentales de problemas: lineales, cuadráticos y no convexos. El proceso experimental comenzó con la obtención de puntos mínimos utilizando bibliotecas establecidas como BilevelJuMP y JuMP. Posteriormente, se generaron problemas estacionarios mediante la adición de componentes aleatorias a estos puntos para cada clase de problema definida.
    
    % Que se comparó
    Los problemas modificados fueron sometidos nuevamente a algoritmos tradicionales implementados en Julia para evaluar su efectividad frente a los puntos estacionarios generados. Se realizó un análisis comparativo destacando los casos más relevantes de cada categoría de puntos estacionarios, considerando la evaluación de la función objetivo del nivel superior en el punto donde se garantizó la estacionariedad, contrastándola con los resultados obtenidos por las bibliotecas convencionales.
    

\end{conclusions}

\begin{recomendations}
        % Recomendaciones
        Como resultado de la investigación realizada, se han identificado varias líneas de trabajo futuro que permitirían expandir y mejorar los resultados obtenidos. En primer lugar, se recomienda ampliar el alcance de la experimentación numérica para incluir una mayor diversidad de problemas de optimización binivel. Esta expansión permitiría validar la robustez y versatilidad del algoritmo propuesto en diferentes contextos y escenarios de aplicación.
    
        En segunda instancia, se sugiere profundizar en la investigación sobre la implementación del algoritmo desarrollado como criterio de parada en nuevos métodos de optimización binivel. Esta línea de investigación podría contribuir significativamente al desarrollo de algoritmos más eficientes y confiables, mejorando la capacidad de detectar y verificar puntos estacionarios durante el proceso de optimización.
        
        Finalmente, se propone el desarrollo de una interfaz gráfica más intuitiva y funcional que facilite la generación automática de puntos según el tipo de estacionariedad requerida. Esta mejora en la usabilidad del software permitiría que usuarios con diferentes niveles de experiencia puedan aprovechar las capacidades del algoritmo de manera más efectiva, ampliando así su aplicabilidad práctica en diversos campos de estudio.
\end{recomendations}

\include{BackMatter/Bibliography}
\begin{appendices}
%%\include{Problems/Cuadraticos/9.2.1}
%\include{Problems/Cuadraticos/9.2.2}
%\include{Problems/Cuadraticos/9.2.3}
%\include{Problems/Cuadraticos/9.2.4}
%\include{Problems/Cuadraticos/9.2.5}

\input{Problems/Cuadraticos/9.2.1}
\input{Problems/Cuadraticos/9.2.2}
\input{Problems/Cuadraticos/9.2.3}
\input{Problems/Cuadraticos/9.2.4}
\input{Problems/Cuadraticos/9.2.5}
%\include{Problems/Cuadraticos/9.2.1}
%\include{Problems/Cuadraticos/9.2.2}
%\include{Problems/Cuadraticos/9.2.3}
%\include{Problems/Cuadraticos/9.2.4}
%\include{Problems/Cuadraticos/9.2.5}

\input{Problems/Cuadraticos/9.2.1}
\input{Problems/Cuadraticos/9.2.2}
\input{Problems/Cuadraticos/9.2.3}
\input{Problems/Cuadraticos/9.2.4}
\input{Problems/Cuadraticos/9.2.5}
%\include{Problems/Cuadraticos/9.2.1}
%\include{Problems/Cuadraticos/9.2.2}
%\include{Problems/Cuadraticos/9.2.3}
%\include{Problems/Cuadraticos/9.2.4}
%\include{Problems/Cuadraticos/9.2.5}

\input{Problems/Cuadraticos/9.2.1}
\input{Problems/Cuadraticos/9.2.2}
\input{Problems/Cuadraticos/9.2.3}
\input{Problems/Cuadraticos/9.2.4}
\input{Problems/Cuadraticos/9.2.5}


\end{appendices}

\end{document}